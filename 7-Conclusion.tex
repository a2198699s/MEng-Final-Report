 \chapter{Conclusion}
\label{conclusion}


% SIL
Safety integrity levels define a standardised approach to the claiming of functional safety for programmable electronic devices. SIL processes were followed closely in the design, implementation and verification of an FPGA-based alternative solution to motor control in an Automatic Transfer Switch. This included following the V-model for ASIC development and by following the processes outlined in the IEC 61508 safety standard.
With the addition of diagnostic components and by performing gate-level and system-level simulations, a safety integrity level could be claimed for this device. Thus allowing the functional safety of the ATyS device to be enhanced and standardised.

% verification
Verification methodologies were adopted to automate and provide metrics for the verification process. UVVM was applied and provided automatic verification of the unitary requirements for each module. OSVVM allowed for the coverage of the simulation to be measured to ensure that the full design had been verified. The verification-suite provided 100\% state coverage and directly tested each of the project requirements for motor control.

% prototyping success
The physical prototype that was developed in this project accurately replicated the behaviour and response of the safety function of the existing microcontroller solution. The FPGA-based ATyS motor control board successfully performed the transfer from one supply of power to another. Additionally, the regulation process in the FPGA-based solution reacted more quickly and performed a smoother regulation of the motor current than in the microcontroller-based system.

% what this means
The company stipulated the requirement for an FPGA-based Automatic Transfer Switch in order to offer a premium product, enhancing their existing family of commercial ATS products. The verified and tested prototype developed for this project replicated the functionality of the existing commercial product and enhanced the functional safety possibilities of the device. 